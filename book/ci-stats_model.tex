\chapter{Statistical Model Computational Infrastructure}

I have implemented statistical model describe above in Python/PyMC,
with Pandas to handle all the data wrangling.  To keep things
manageable, I have broken the model into consistuient elements,
corresponding to the theoretical elements developed in the sections of
Chapter~\ref{TK}.  These are the age-pattern model, the covariate
model, the age-integrating model, the data model, and the consistency
model.


\section{Age-pattern Model}

The age pattern model implements a non-negative, piecewise linear
Gaussian process $\boldmu$ as an exponentiated piecewise-constant spline
with spline effects $\gamma_1,\ldots,\gamma_K$ and knots
$a_1,\ldots,a_K$.  The \Matern covariance function is included by
giving $(\gamma_k)_{k=1}^K$ a multivariate Normal prior distribution
with the appropriate variance-covariance function (represented as a
Cholesky factored $k\times k$ matrix for computational efficiency).

This should be checked to confirm that adding knots does not change
the level of smoothing, which can be done theoretically or
experimentally.


\section{Age-integrating Model}

The age-integrating model maps the age pattern to age intervals,
according to the data.  There are several approach to this that I
would like to explore, and the simplest is to approximate the integral
of the age pattern from $a_0$ to $a_1$ by the midpoint,
\[
\mu_i = \int_{a_{0,i}}^_{a_{1,i}} \boldmu(a) d\boldw_i(a) \approx \boldmu\left(\frac{a_0+a_1}{2}\right).
\]

I plan to demonstrate that this is not appropriate for much of the
data we deal with, but it is a good starting place to make sure
everything works together, and it will provide a way to assess the
cost/accuracy tradeoff of more precise approximations to the integral.





\section{Covariate Model}

The covariate model 
