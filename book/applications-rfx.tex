\chapter{Dealing with geographical variation: hepatitis C}
\label{applications-rfx}

Hepatitis C is a viral infection that attacks the liver.  In a small
portion of acute cases, the body can eliminate the virus; however the
majority of acute cases develop into chronic infections.  Chronic
infections cause liver damage and may develop into end stage liver
disease or cirrhosis.  Few, if any, chronic cases experience symptoms
and only one third of acute cases are symptomatic and jaundice.
Chronic symptoms are nonspecific, intermittent and mild with the most
common symptom being fatigue.  Common symptoms for severe and advanced
disease stages include nausea, dark urine and jaundice.  Since
hepatitis C infections are asymptomatic, diagnosis requires laboratory
testing for both hepatitis antibodies (anti-HCV) and the hepatitis
virus (HCV RNA).  There is no vaccination for hepatitis C, but therapy
can prevent advanced liver disease. \cite{hoofnagle_hepatitis_1997,
  ghany_diagnosis_2009, hanafiah_global_2012}

Compared to other countries in the region, Egypt has a high hepatitis
C prevalence.  In an attempt to treat endemic schistosomiasis, a
common parasitic worm that affects the urinary tract, gut and liver,
the Egyptian Ministry of Health launched widespread injection-based
treatment throughout 1950-1980.  While there were improvements in
schistosomiasis-induced mortality, recycled needles and poor needle
sterilization infected many with hepatitis C. \cite{frank_role_2000,
  mezban_hepatitis_2006, strickland_liver_2006} The spatial variations
of hepatitis C in North Africa and the Middle East make it an
excellent example for hierarchical random effects modeling.

Random effects modeling detects systematic differences among different
hierarchies, or levels, of data.  The spatial hierarchy in the GBD
2010 study uses countries nested in regions nested in super-regions.
There are 21 regions defined by demographic and epidemiological
similarities that are further clustered by 7 super-regions.

The analysis of hepatitis C uses data on the prevalence of persons who
have hepatitis C antibodies.  Incomplete data or data from high-risk
populations, such as health care workers, were excluded.  Notice that
hepatitis C prevalence in Egypt is more than 40 times that of the
Jordan, even though they are both in the region of Northern Africa and
the Middle East (Figure \ref{fig:app-hepc data}).

    \begin{figure}[h]
        \begin{center}
            \includegraphics[width=\textwidth]{hepc-data_EGY_v_JOR.pdf}
            \caption{Prevalence data from systematic review of
              hepatitis C in Jordan (panel (a)) and Egypt (panel
              (b)).}
            \label{fig:app-hepc data}
        \end{center}
    \end{figure}

The analysis uses an age-standardizing hierarchical random effects
generalized negative binomial spline model to estimate prevalence.
The hierarchical random effects allow the model to capture variation
within the region of North Africa and the Middle East.  Looking at
Table \ref{tab:app-hepc regional rfx}, Egypt (EGY) has significantly
higher prevalence than the other countries in the region.  Figure
\ref{fig:app-hepc regional rfx} confirms this as the prevalence
estimate for Egypt is much above the regional average.

    \begin{table}[h]
        \begin{center}
        \caption{ Estimates of the intercept shift of hepatitis C prevalence in log space from a random effects model in the region of North Africa and the Middle East.}
        \label{tab:app-hepc regional rfx}
        \begin{tabular}{|c|c|c|c|}
            \hline
                Country & Posterior Mean & Lower 95\% HPD  & Upper 95\%  HPD \\
            \hline
                EGY	&	1.87	&	 1.5	&	2.2	\\
                JOR	&	-0.59	&	-1.1	&	-0.1 \\
                SAU	&	-0.78	&	-1.2	&	-0.3 \\
                IRQ	&	0.06	&	-0.4	&	0.6	\\
                IRN	&	0.02	&	-0.5	&	0.5	\\
                YEM	&	0.05	&	-0.4	&	0.5	\\
                TUR	&	-0.32	&	-0.7	&	0.0	\\
                SYR	&	-0.14	&	-0.6	&	0.3	\\
                TUN	&	-0.19	&	-0.6	&	0.3	\\
            \hline
        \end{tabular}
        \end{center}
    \end{table}

    \begin{figure}[h]
        \begin{center}
            \includegraphics[width=\textwidth]{hepc-region_v_EGY_v_JOR.pdf}
            \caption{The 1990 estimate of hepatitis C prevalence for men in the region of North Africa and Middle East and the countries Egypt and Jordan.  These estimates only use 2 levels in the hierarchal random effects model--region and country.}
            \label{fig:app-hepc regional rfx}
        \end{center}
    \end{figure}

In such noisy data, placing a prior on the dispersion of the data
informs the model of the data heterogeneity.  This allows the model to
infer how dispersed the random effects are between geographic regions,
hence quantifying the uncertainty in the geographic regions for which
no data are available.  Priors on the dispersion parameter, $\delta$,
may be one of three categories, `very', `moderately' or `slightly'.
The natural logarithm of $\delta$ is uniformly distributed between its
lower and upper bounds.  Intended as a weakly informative prior, the
bounds of the categories overlap, so that the bounds of 'very' are
[1,9], 'moderately' are [3,27] and 'slightly' are [9,81].

In this example, the effects of priors on the overdispersion of
$\delta$ are seen in the posterior estimates at the country level as
seen in Figure \ref{fig:app-hepc global hetero}.  Random effects
modeling detects within sample variation and true variation that
cannot be explained by a covariate.  Therefore, a change in the prior
on global heterogeneity changes the level of variation and thus the
random effect size.  As seen in figure \ref{fig:app-hepc global
  hetero}, when the prior on global heterogeneity is `very', the
estimates are compressed.

    \begin{figure}[h]
        \begin{center}
            \includegraphics[width=\textwidth]{hepc-tree_plot_global_hetero.pdf}
            \caption{The 1990 intercept shift of hepatitis C
              prevalence in log space for men with different priors on
              global heterogeneity, $\delta$.  Four levels (global,
              super-region, region, country) were used in the
              hierarchal random effects model.}
            \label{fig:app-hepc global hetero}
        \end{center}
    \end{figure}

Another way to view compressed estimates is by looking at the
age-standardized prevalence in Table \ref{tab:app-hepc global rfx}.
As heterogeneity increases from `slightly' to `very', country
estimates are compressed toward the regional mean.

    \begin{table}[h]
        \begin{center}
        \caption{ Hepatitis C age-standardized prevalence estimates from a hierarchal random effects single rate type model with differing priors on global heterogeneity.}
        \label{tab:app-hepc global rfx}
        \begin{tabular}{|c|c|c|c|}
            \hline
                Geographic Area & Heterogeneity & Posterior Mean & Standard Deviation \\
            \hline
                North Africa Middle East & $\delta = [9,81]$ & 0.049 & 0.005 \\
                & $\delta = [3,27]$ & 0.049 & 0.005 \\
                & $\delta = [1,9]$ & 0.045 & 0.006 \\
            \hline
                Jordan & $\delta = [9,81]$ & 0.007 & 0.002 \\
                & $\delta = [3,27]$ & 0.010 & 0.003 \\
                & $\delta = [1,9]$ & 0.020 & 0.007 \\
            \hline
                Egypt & $\delta = [9,81]$ & 0.189 & 0.019 \\
                & $\delta = [3,27]$ & 0.179 & 0.023 \\
                & $\delta = [1,9]$ & 0.137 & 0.022 \\
            \hline
        \end{tabular}
        \end{center}
    \end{table}
