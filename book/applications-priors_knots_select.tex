\chapter{Unclear age pattern, requiring expert priors: premenstrual syndrome}
\label{applications-priors_knots_select}

Epidemiological data without clear age-patterns are a reoccurring
theme in the GBD 2010 study.  Unclear age-patterns make expert priors
essential in the modeling process.  However, such cases are very
sensitive to the choice of prior assumptions, as shown in the
following example of premenstrual syndrome (PMS) in Western Europe.

PMS is a common cyclic disorder that affects women of reproductive
years during the period between ovulation and the onset of menses.
More than 200 behavioral, psychologic and physical symptoms have been
associated with PMS, the most common being irritability, tension,
depression, bloating, weight gain and food cravings.  There is no
known cause or consistent
treatment. \cite{dickerson_premenstrual_2003, singh_incidence_1998,
  goodale_alleviation_1990}

A meta-analysis of data from a systematic review on the descriptive
epidemiology of premenstrual syndrome yielded 74 rows of prevalence
data, of which 18 were from Western Europe.  As seen from Figure
\ref{fig:app-pms_data}, the data are noisy, with overlapping and
heterogeneous age groups that show no clear age pattern.

    \begin{figure}[h]
        \begin{center}
            \includegraphics[width=\textwidth]{pms-data.pdf}
            \caption{Prevalence data for women with premenstrual
              syndrome in Western Europe.}
        \end{center}
        \label{fig:app-pms_data}
    \end{figure}

\section{Priors on level} \label{sec:app-priors on level}
With no clear guidance from the data, informative priors to identify
an age pattern are a critical part of the modeling process, even
though they may have unintended effects as discussed in Chapter
\ref{theory-age_pattern_model}.  To illustrate the effects of priors,
a single rate type model is used without invoking the compartmental
model from Chapter \ref{sys-dynamics}.

Looking at the data in Figure \ref{fig:app-pms_data}, no data exists
before age 15 or after age 50.  Since PMS is a disorder related to the
cycles of the female reproductive system, it is obvious that the data
outside this age range are not present for biological reasons.
However, this information is unknown to the model and without a prior
limiting the age range to 15-60, the single rate type model estimates
prevalence for the entire age range, as seen in Figure
\ref{fig:app-prios_on_level}.  Here expert knowledge needs to inform
the model that prevalence should not be expected outside of the ages
15-50.

    \begin{figure}
        \begin{center}
            \includegraphics[width=\textwidth]{pms-priors.pdf}
        \end{center}
        \caption{PMS is a disorder related to the cycles of the female
          reproductive system.  As shown in panel (a), without a level
          prior to inform the model that prevalence data is not
          present outside of the ages 15-50 for biological reasons, it
          will create prevalence estimates for all ages, regardless of
          biological feasibility.  Restricting prevalence to above age
          15 in panel (b), below age 50 in panel (c), or between ages
          15-50 in panel (d) changes the prevalence estimates
          dramatically for women in Western Europe with PMS.}
        \label{fig:app-prios_on_level}
    \end{figure}

\section{Knot location}
As previously discussed in Chapters
\ref{theory-age_group_model-overlapping_data} and
\ref{applications-splines_knot_loc}, age-specific hazards are modeled
with splines, using knots to partition the age range into intervals.
Models will not be very sensitive to choice of knots with ample data
and clear age patterns.  However, with sparse and noisy data without a
clear age pattern, the number and location of knots can influence the
model results substantially as seen in Figure \ref{fig:app-knot_loc}.
Choosing the number and location of knots a priori using expert
knowledge allows the user to determine critical features of the model.

    \begin{figure}
        \begin{center}
            \includegraphics[width=\textwidth]{pms-knot_location.pdf}
        \end{center}
        \caption{All panels have knots at \{0, 15, 50, 100\} and vary
          the number and location of knots between the ages of 15 and
          50 to show the sensitivity of knot selection sparse and
          noisy data without a clear age pattern. Even with 1 knot,
          the placement at age 20, 32 or 45 gives markedly different
          estimates of PMS prevalence in Western Europe (panel (a)).
          Panel (b) uses 2 knots and varies their locations at \{27,
          38\}, \{20, 45\}, or \{30, 35\} while panel (c) uses 3 knots
          at locations \{23, 32, 41\}, \{18, 32, 47\} and \{29, 32,
          35\}.}
        \label{fig:app-knot_loc}
    \end{figure}

\section{Priors on monotonicity}
Another common prior for age patterns is the belief that the
epidemiologic parameter increases or decreases over a certain age
range.  As seen in Figure \ref{fig:app-knot_loc}, priors on
monotonicity between the critical ages of 25 and 40 have a large
effect on the prevalence estimate for Western Europe.

    \begin{figure}
        \begin{center}
            \includegraphics[width=\textwidth]{pms-direction.pdf}
        \end{center}
        \caption{Between the ages of 25-40, the prior on monotonicity
          makes a large impact on the prevalence estimates for women
          in Western Europe with premenstrual syndrome.}
        \label{fig:app-knot_loc}
    \end{figure} 
