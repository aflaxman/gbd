\chapter{Statistical models for heterogeneous age groups}
With a full development of statistical rate models for a single age
group behind us, this section turns to a peculiar feature of
population rate meta-analysis: the wide variety of age groups reported
in the literature.

A typical example of the heterogeneity in age groups is shown in for
the systematic review results on atrial fibrilation (AF) in
Figure~\ref{age-group-model-af-age-groups}.  The midpoint of the age
group is scattered against the width of the age group.  Simply put,
there is no standard set of age groups for AF research, and different
studies report results with different age groups. Unfortunately, this
phenomenon is far from unique to AF.

\begin{figure}[h]
\begin{center}
\includegraphics[width=\textwidth]{af_age_groups_scatter.pdf}
%\includegraphics[width=\textwidth]{af_ages_intervals.pdf}
\end{center}
\caption{Mean and spread of age groups in the prevalence data
  collected from a systematic review of atrial fibrilation. The
  size of the circle shows how many observations of this age group
  were found in systematic review. There were
  $586$ rows of prevalence data
  extracted, but the most common age group accounted for only
  $68$ rows.}
\label{age-group-model-af-age-groups}
\end{figure}

This variation in reporting would not be problematic if I had access
to the microdata from all of the systematic review studies.  For
example, using microdata from a national health information system or
from a demographic household survey, I could simply tally the
prevalence rates by single-year age groups.  Although each individual
rate gathered in this way would have high variability, the rate model
from Chapter~\ref{theory-rate_model} combined with the spline-based
age pattern model from Chapter~\ref{theory-age_pattern_model} would
work to produce an estimate that is as uncertain as it should be.

Re-analysis from microdata is occasionally implemented in a GBD study,
however it is often not an option.  I expect the use of microdata
re-analysis to become more frequent in national and subnational
settings.  In the more common situation where rate microdata are not
available, the rates cannot be retallied into homogeneous age groups,
and an alternative approach is needed.

There are several statistical approaches which I have considered, and
they will be compared and contrasted in this section.  Before getting
into the details, however, it is worthwhile to examine theoretically
the way that age grouping functions.

I begin with a simple mechanistic model of the age grouping process.
A study conducts some sort of measurement on a population of
individuals who are all of different ages and then the
epidemiological rate or rates of interest are tallied for age groups
selected in some context dependent manner. If the study was a
prevalence study using a full census sample, for example, and if I use
$r_{a_0,a_1}$ to denote the rate for age group $(a_0, a_1)$ and
$n_{a_0,a_1}$ to denote the subpopulation size of age group
$(a_0,a_1)$, then the identity
\[
n_{a_0, a_2} = n_{a_0,a_1} + n_{a_1,a_2}
\]
says nothing more complicated than that the size of the subpopulation
of age at least $a_0$ and less than $a_2$ is the sum of the size of
the subpopulation between age $a_0$ and $a_1$ and the size of the
subpopulation between age $a_1$ and $a_2$.  Applying the same
observation to the part of these subpopulations that have the condition
of interest yields the following identity:
\[
r_{a_0,a_2} = r_{a_0,a_1}\frac{n_{a_0,a_1}}{n_{a_0,a_2}} + r_{a_1,a_2}\frac{n_{a_1,a_2}}{n_{a_0,a_2}}. 
\] 
In a limiting case of a very large population with very fine age
intervals, this becomes:
\[
r_{a_0,a_2} = \int_{a=a_0}^{a_2} r_{a,a+\d a}\frac{n_{a,a+\d a}}{n_{a_0,a_2}}\d a.
\]
Undoubtedly all real studies are more complicated than this full
census of prevalence, but this is a starting point for conceptualizing
where age-grouped rates come from.  Roughly, they are integrals over
instantaneous rates for infinitesimal age groups.

\section{Overlapping age group data}
\label{theory-age_group_model-overlapping_data}
This section explores an examples of overlapping age group data
collected in systematic review through graphical statistics.  The
primary way I like to display overlapping age group data is shown in
Figure~\ref{theory-age_group_model-dismod_data_plot} as horizontal
lines on a plot of age versus rate value.  The level of the bars shows
the rate value, while the width of the bars shows the range of ages
included in the age group. It is often informative to augment these
lines with error bars, showing the uncertainty reported for each rate
value, but for this section I have left out the representation of
uncertainty to keep the plots as simple as possible.

\begin{figure}[ht]
\begin{center}
\includegraphics[width=\textwidth]{epilepsy_ages_intervals.pdf}
\caption{The systematic review of the descriptive epidemiology of
  epilepsy included $79$ observations of disease prevalence for the United States
  with start year before 2000. The prevalence level and age group of
  each observation is shown above as a horizontal bar, with the
  position of the bar along the y-axis representing the prevalence
  level and the endpoints along the x-axis representing the start and
  end of the age group.  The data shows heterogeneity by age that is
  typical for these systematic review results, appearing to increase
  slightly with age from the teens through adulthood but with an unclear trend
  at younger and older ages.  }
\label{theory-age_group_model-dismod_data_plot}
\end{center}
\end{figure}

Each of the horizontal lines in
Figure~\ref{theory-age_group_model-dismod_data_plot} can be
represented as a triple $({a_s}, {a_e}, r)$, where $a_s$ is the
starting age of the age group, $a_e$ is the ending age of the age
group, and $r$ is the rate observed for this age group.

A brief word about ${a_e}$ is in order here.  Often in the
epidemiological literature, the ending ages are described in a
unit-dependent fashion, for example age group 10-14.  This is intended
to mean from the first day of age 10 to the last day of age 14.
However, this notation can be a hindrance when dealing with age
resolution finer than one year, a situation that comes up when studying
neonatal conditions.  For this reason, I prefer the approach that
takes the end age of the interval to be the first age where an
individual is no longer part of the group.  In the case above, I would
say ${a_e} = 15$.


With a firm understanding of the sort of overlapping age group data
that arises in systematic review, I now turn to developing and
analyzing a series of models for the meta-analysis of the data.  There
are five that I will consider: the midpoint model, the disaggregation
model, the midpoint-with-covariate model, the age-standardizing model,
and the age-integrating model.  The age-standardizing model is the
balance of theoretical foundations, practical implementability, and
empirical success that is used in the second half of this
book.

\section{Midpoint model}

The simplest approach to modeling data with heterogeneous age
intervals, is to apply each rate measurement to the midpoint of the
age group it measured.  This is trivial operationally, but it is also
theoretically justified through a ``trapazoidal rule'' integration.

In practice, this approach is quite accurate for modeling a
disease rate that changes slowly as a function of age.  However, it
becomes inaccurate when modeling rates that change more
rapidly.  The typical setting in applications in the second half of
this book will include a few studies that focus on age patterns and
hence have narrow age groups, together with many other studies
that focus on other aspects of disease epidemiology.  Thus the
relevant setting to consider how these models are inaccurate is where
there are a few small age group studies and many large age group
studies.

Mathematically, the formulation is as follows: let $\boldmu(a)$ be a process-model
for the age-specific rate (e.g a spline model from
Chapter~\ref{theory-age_pattern_model}), and let $\dens(r,n\given \pi,
\rho)$ be a data-model for the observed level (e.g the negative-binomial rate model from
Chapter~\ref{theory-rate_model}).  Then the likelihood of an
observation of rate $r_i$ with effective sample size $n_i$ for age
group $({a_s}_i, {a_e}_i)$ is simply $\dens\left(r_i, n_i \given
\boldmu\left(\frac{{a_s}_i+{a_e}_i}{2}\right),
\rho\right)$. Equivalently, in ``blackboard notation,'' using
$\scD(\pi, \rho; n_i)$ to denote the rate model
distribution, I can write
\begin{align*}
r_i &\sim \scD\left(\boldmu(a_i), \rho; n_i\right),\\
a_i &= \frac{{a_s}_i+{a_e}_i}{2}.
\end{align*}
This formulation will be convenient for comparison with the other models of age groups to come.

Figure~\ref{midpoint} compares the estimate produced by the midpoint
model to ground truth through simulation in two settings.  When the
age-specific rate varies little as a function of age, as shown in
panel (a), the estimated rate is quite accurate.  But when the
age-specific rate varies substantially, as shown in panel (b), the
estimate is biased.


\begin{figure}[h]
\begin{center}
\includegraphics[width=\textwidth]{age_group_midpoint.pdf}
\caption{The midpoint model, a conceptually simple approach to
  dealing with data with heterogeneous age groups, simply
  attributes the observation to the midpoint of the age group.  Panel
  (a) shows the model applied to an age-specific rate that does not
  vary a great deal across ages for which the midpoint model is a
  better fit.  Panel (b) shows the model applied to an age-specific rate
  that varies more for which the midpoint model overcompresses the
  estimates.}
\label{midpoint}
\end{center}
\end{figure}


\section{Disaggregation model}
An alternative to the midpoint model which seems appealing but has
some downsides is what I call \emph{disaggregation}.  To understand
the disaggregation approach, imagine the simple re-analysis that I
could do if microdata were available (as described at the beginning of
this chapter).  If I had access to the individual measurements that
went into the calculation of the disease rate found in systematic
review, I could do a re-analysis with any age grouping I wished. I
could calculate rates for single-year age groups, and be sure that the
age pattern is not changing substantially during the grouping.

The microdata from rates found in systematic review are rarely
available, however. The disaggregation approach is a simple attempt to
impute what the rates for the desired age grouping would be \emph{if}
the microdata were available. This requires taking into account the
increased variation that would be found if a study of the same size
was reported for finer age groups.

Without any additional information, rate data reporting a level of $r$
for a population with effective sample size $n$ for age group $(a_s,a_e)$, i.e.,
\[
X = (r, n, a_s, a_e)
\]
can be disaggregated into $A = a_e-a_s$ rows of
data, $X_1, X_2, \ldots, X_A$, with 
\[ 
X_a = \left(r, \frac{n}{a_e-a_s}, a, a+1\right), \text{for } a=1,2,\ldots,A. 
\]

Disaggregation can be interpreted as a data preprocessing step, and
this disaggregated data can be fed to the midpoint model from the
previous section to produce a comprehensive estimate of the rate as a
function of age. However, this model has some unintended negative
features when large age intervals are disaggregated.  Because it
ignores the correlation in age of disease levels, it tends to
overcompress age patterns at young and old ages, as shown in Figure~\ref{disagg}.

\begin{figure}[h]
\begin{center}
\includegraphics[width=\textwidth]{age_group_disagg.pdf}
\caption{This figure shows the effects of fitting a model with this
  disaggregation approach to two simulated data sets.  When the age groups are sufficiently
  fine-grained and homogeneous, disaggregation is a successful
  approach.  But with even slight heterogeneity, as in panel (b), the model
  estimates are overcompressed}
\label{disagg}
\end{center}
\end{figure}


\section{Midpoint model with group width covariate}
An alternative method, which I consider more ``statistical'' in its
approach, is to add the width of the age group as a covariate into the
midpoint model.  This model takes the form
\begin{align*}
r_i &\sim \scD\left(\mu_i, \rho; n_i\right),\\
\mu_i &= \boldmu\left(\frac{{a_s}_i+{a_e}_i}{2}\right) + \theta (a_e - a_s).
\end{align*}

This addresses the shortcomings of the disaggregation approach
\emph{indirectly}, and the indirect nature has positives and
negatives.  This method does not explicitly connect the large age
interval to the small age interval but instead allows the data to
inform the relationship.  On the other hand, it posits that the
data-driven relationship between the rates for studies with the same
midpoint but different age groups is a linear relationship. In
contrast, the mathematical model developed at the beginning of this
chapter is nonlinear in
a specific and mechanistically known way.
Figure~\ref{midpoint-covariate} shows the results of applying the midpoint-covariate model
model to simulated data.


\begin{figure}[h]
\begin{center}
\includegraphics[width=\textwidth]{age_group_midpoint_covariate.pdf}
\caption{The midpoint-covariate model applied to two simulated
  datasets, where ground truth is known. Although this approach is
  appealing theoretically, the added flexibility of the covariate
  model performs poorly in practice. }
\label{midpoint-covariate}
\end{center}
\end{figure}

\section{Age-standardizing and age-averaging models}
An even more complicated approach, both conceptually and
computationally, is to average across the age interval explicitly in
the statistical model:
\begin{align*}
r_i &\sim \scD\left(\mu_i, \rho; n_i\right),\\
\mu_i &= \int_{a={a_s}_i}^{{a_e}_i} \boldmu(a)\d w_i(a),
\end{align*}
where the integration $\d w_i$ is weighted according to population
structure.

This has the theoretical appeal of matching the generative model above
but the drawback of being slower to compute and less stable
numerically.  It also has a major piece left unspecified, the
selection of the age weights for the integration.  There are two
sensible approaches to this, which I call the \emph{age-standardizing
  model} and the \emph{age-averaging model}.  The age-standardizing
model uses a common age pattern $\d w_i(a) = \d w(a)$ for all studies, while
the age-averaging model uses the best estimate available of the age
pattern of the study population in each observation.  The
age-standardizing model is faster, due to a computational optimization
only possible when the $\d w_i$ are the same for all $i$, but the
age-averaging model is appealing on theoretical grounds, because it
can make use the most information.  However, it is not certain that
using this information will make the end results any more accurate,
because the age pattern of the study population is rarely known with
much certainty, and often it is necessary to assume that it matches
the national age pattern for the country-years where the study was
conducted.  In the case of remission and mortality studies it is even
more complicated to estimate the study population age pattern, since
it is \emph{not} the same as the national population age pattern, but
modulated by the age pattern of disease prevalence.
Figure~\ref{age-group-standardize} shows the results of this model on
simulated data.

\begin{figure}[h]
\begin{center}
\includegraphics[width=\textwidth]{age_group_standardize.pdf}
\caption{The age-standardizing model applied to simulated data with a
  known age-specific rate function as ground truth.  The results in
  panel (a) show that the model 
  recovers the true age pattern quite precisely. Panel (b) shows that the
  results are still accurate when the data generation procedure is
  even more noisy.}
\label{age-group-standardize}
\end{center}
\end{figure}


\section{Model comparison}


\begin{figure}[h]
\begin{center}
\includegraphics[width=\textwidth]{age_group_models.pdf}
\caption{A comparison of $4$ models for heterogeneous age groups, showing that the age-standardizing model comes closest to recovering the truth.  This corresponds to the results of the simulation study presented below in Table~\ref{age_group_comparison}.}
\label{age-group-model-comparison}
\end{center}
\end{figure}


This section provides a comparison of the approaches to age group
modeling.

An appropriate comparison of these approaches is somewhat difficult to
develop.  One approach is through simulation study, where a dataset is
simulated from known ground truth.  This allows the estimates to be
compared to ``true'' values, but this risks
inappropriate model selection due to inaccurately choosing the
distribution of the simulated data.  Another approach is
cross-validation, where data from systematic review is split into
mutually exculsive
\emph{training} and \emph{test} sets, and the model is fit to the
training set and used to predict the values in the test set.  Naively
holding out $25\%$ of the data doesn't address the exact topic of
interest, however, since it determines which model predicts rates of
all age groups, and I am really only interested in predicting the age
groups with small widths accurately.  It would be preferable to hold
out only data with small-width age groups from large representative
subpopulations.  Unfortunately there is rarely enough data to do this,
especially in all the settings that come up in disease modeling.

I have taken a pragmatic approach, evaluating with a natural
simulation described below.  Future work, based on more sophisticated
simulation scenarios or based on carefully designed hold-out
cross-validation, is necessary to further understand the trade-offs
between these alternative methods.

The data simulation procedure I used is the following:
\begin{itemize}
\item Choose age intervals for $30$ rows of data; for $i=1,\ldots,10$,
  $({a_s}_i,{a_e}_i) = (10(i-1), 10i)$, and for the remaining $20$
  intervals, choose the age interval width uniformly at random from $[1,100]$
  and choose the midpoint of the age interval uniformly at random from ages
  which admit this age range.

\item Choose the effective sample size $n_i$ for each row, uniformly at random from $[10^2, 10^4]$.

\item Choose an age-specific population structure for each row of data,
  with the form $w_i(a) = e^{\beta_i a}$, where $\beta_i$ is drawn
  from a normal distribution with mean $0$ and standard deviation
  $\frac{1}{10}$.

\item Calculate the true rate value for each age interval,
  $r^\text{true}_i = \sum_{a={a_s}_i}^{{a_e}_i} \pi_\text{true}(a)
  w_i(a)$, where $\pi_\text{true}(a) =
  \exp\left(\frac{3(a-35)^2}{1000} + \frac{a-35}{100}\right).$

\item Choose an observed rate value, based on a negative binomial distribution:
$r_in_i \sim \NegativeBinomial(r^\text{true}_i, \delta_\text{true})$, where $\delta_\text{true} = 5$.
\end{itemize}

Table~\ref{age_group_comparison} shows the median results of fitting this simulated data with a variety of models.

\begin{table}

\begin{center}
\begin{tabular}{|c|c|c|c|c|}
\hline
model&bias&mae&pc&time\\
\hline
midpoint&0.02&0.08&0.8&29.0\\
disaggregation&0.03&0.19&0.08&52.6\\
midpoint-covariate&0.03&0.09&0.89&45.8\\
age-standardizing&0.01&0.04&0.95&30.3\\
age-integrating&-0.0&0.03&0.78&38.1\\
\hline
\end{tabular}
\end{center}

\caption{Median results for $100$ replicates of the simulation study
  comparing age-specific rate estimates from $5$ models of age-grouped
  data, showing bias (mean of true minus predicted), median absolute
  error (mae, median of absolute difference between truth and
  predicted), probability coverage (pc, fraction of truth falling
  within 95\% uncertainty interval of prediction), and computation time. The
  age-standardizing and age-integrating models are superior in all
  metrics of fit quality.  The age-standardizing model has computation time only slightly more
  than the fastest approach, while the age-integrating model is 26\% slower.}
\label{age_group_comparison}
\end{table}
