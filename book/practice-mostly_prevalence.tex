\section{Modeling diseases with mostly prevalence data}

A surprising headline of the first GBD study was the unexpectedly
large burden of mental disorders like depression and
schizophrenia. The epidemiology of these conditions, as well as many
other mental and neurological disorders, are known primarily from
prevalence studies. This chapter considers in detail the setting where
prevalence data is the primary source of input data.

Several gynecological disorders had systematic reviews which resulted
in only a small amount of prevalence data, and no other measurements
of epidemiological rates like incidence, remission, or mortality.  (TK
really no studies on remission?)

TK A lot of prevalence, expert priors on everything else

TK Prevalence plus a few data points on other things

TK Prevalence data which clearly violates time equilibrium assumption

TK Birth prevalence data only -- no need for fancy statistical model

\subsection{PMS}
\begin{figure}
\begin{center}
\includegraphics[width=\textwidth]{pms-prev.pdf}
\end{center}
\caption{(a) All prevalence data collected in systematic review; (b)
  Prevalance data for North America, High Income region only.  Note
  the extremely high heterogeneity of the data, ranging from
  $<<d['pms.json|dexy']['min_rate_per_100']>>$ per $100$ to
  $<<d['pms.json|dexy']['min_rate_per_100']>>$ per $100$.  Covariate
  modeling can help explain this variation.}
\label{pms-prev}
\end{figure}

Figure~\ref{pms-prev}a shows all of the data collected in systematic
review.  Since the review found only data on disease prevalence, and
even that is sparse (available for only TK regions) and extremely
noisy, choices made in the modeling process will have a significant
effect on the resulting estimates.  This highlights the importance of
making it clear what the assumptions are, and how sensitive the
results are to these assumptions.

The data gathered in systematic review of premenstural syndrome show
just how sever the situation is: if you come up with a number
between 0 and 1, you could find a study where the prevalence is within 5
per 100.  Even restricting our attention to data from the North
America, High Income GBD region (USA and Canada) still yields a data set
with no clear age pattern (horizontal bars in Figure~\ref{pms-prev}b).

The prudent solution in this situation would be to give up, and not
even produce estimates of PMS prevalence, let alone incidence and
duration.  For a well-funded researcher who really must know the
answer, proper recourse would be to field a study designed to measure
exactly the quantity of interest in exactly the population of
interest.  Since all of the wildly varying data collected in
systematic review shows that the prevalence is at least
$<<d['pms.json|dexy']['min_rate_per_100']>>$ per $100$, this study
need not be enormous.

However, we cannot opt out of estimating and we cannot wait for a new
definitive study to be conducted (although we can use the results of
the systematic review to decide which studies are highest priority in
the future).  So model-based estimates are our only option.

The simplifying assumptions that experts in these diseases have agreed
to are the following: there no incidence or prevalence of disease
before age 15 or after age 50, there is no excess mortality, and there
is no remission before age 40.  These assumptions correspond breaks
and midpoints of the age groups to be estimated in the GBD 2010 study,
which include groups for 10-15, 15-20, 20-25, 25-35, 35-45, and 45-55.

With these restrictions to the age patterns in place, the consistent
fit of all of the world's pooled data is shown in Figure~\ref{pms-consistent}.
\begin{figure}
\begin{center}
%\includegraphics[width=\textwidth]{pms-consistent.pdf}
TK Graphics showing mortality, incidence, remission, and prevalence
\end{center}
\caption{Consistent fit of all prevalence data collected in systematic
  review together with expert priors on age pattern.  The expert
  priors are: no incidence or prevalence of disease before age 15 or
  after age 50, there is no excess mortality, and there is no
  remission before age 40.  Knots in the piecewise constant Gaussian
  process model evenly spaced at 5 year intervals.}
\label{pms-consistent}
\end{figure}

This model is sensitive to the coarse knot selection in the age
pattern model, which must be chosen finely enough to allow consistent
fits that respect the expert priors on when age-specific rates are
non-zero.  Choosing an age pattern model with knots evenly spaced in
5-year intervals is compatible with the model assumptions that there
are changes between zero and non-zero rates at ages 15, 40, and 50.
Knots forming a coarser mesh could speed computation time, but should similarly be
choosen to be compatible with the zero-rate assumptions.  For example,
taking knots at $\{0,15,40,50,100\}$ is the coarsest mesh that could
possibly respect the zero-rate assumptions.  On the other hand, taking
knots on a finer mesh gives the model more flexibility in representing
the age pattern (which might not be a good thing, considering how
noisy the data appears).  Figure~\ref{pms-grids} compares the effects
of fitting the data from North America, High Income to finer and coarses grids.
\begin{figure}
\begin{center}
%\includegraphics[width=\textwidth]{pms-grids.pdf}
TK Graphics showing prevalence for NAHI on coarse, medium, and fine grids
\end{center}
\caption{Knots in the age pattern models with different patterns. (a)
  coarsest grid at $\{0,15,40,50,100\}$; (b) coarse grid at
  $\{0,15,25,35,40,50,100\}$;
  (c) fine grid at 5-year intervals; (d) fine grid at 2-year intervals.}
\label{pms-grids}
\end{figure}

Although choosing between these knots in the model can be based on
expert judgement, there are objective measures of model fit, discussed
in Section~\ref{TK}, such as in-sample or out-of-sample predictive
accuracy and graphical checks of the posterior predictive
distribution.  Figure~\ref{pms-ppc} shows the the posterior predictive distributions for all of
the data from NAHI in a model with no covariates.  In this model, the
posterior distribution for the over-dispersion parameter has median
value TK.

This provides a reasonable starting point, but TK discussion of
over-dispersion posterior and posterior predictive checks of data.  We
can potentially do better by using covariates to explain some of the
variation seen in the data.  TK discussion of relevant study-level
covariates.  TK discussion of the existance or lack thereof of
appropriate country-level covariates, such as health system access.

\begin{figure}
\begin{center}
%\includegraphics[width=\textwidth]{pms-ppc.pdf}
TK Graphics posterior predictive check for NAHI with and without
study-level covariates
\end{center}
\caption{Posterior predictive checks for NAHI with and without
  study-level covariates.}
\label{pms-ppc}
\end{figure}

TK discussion of a demonstration of how including these covariates
results in less dispersed posterior predictive distributions, and a
comparison of the posterior values of the over-dispersion terms in
both.  Also appropriate here to look at the BIC and DIC values.  AIC
is not appropriate for reasons to be stated.

TK discussion of producing estimates that differ by region and  by time
using an empirical bayes approach, with the world estimate as the
empirical prior.

TK discussion of the possibility of an acceptible result, although it
is important to investigate the effects of smoothing priors,
heterogeneity priors, level bounds.

TK discussion of when to start smoothing priors, e.g. a fine age mesh
together with a smoothing prior that starts after menarch, leads to
very different age pattern, and possibly different levels overall.

\subsection{Cannabis Dependence, also use}
14390 15303 

16153 16160 

\subsection{Bipolar}
The systematic review for epidemiological rates related to bipolar
disorder came up with TK rows of prevalence data, and TK rows of
standardized mortality ratio (SMR) data.  Since there is so much more
prevalence data than SMR data, I have pooled all of the rows of SMR
data across studies and applied them assuming that SMR for bipolar
does not vary by region or time.  There may be sufficient data to assume
that it \emph{does} vary by sex, however. 

The approach I followed in this setting is one that has come up quite
frequently.  Generate an empirical prior on prevalence, by pooling all
of the world's data in a model with effects for region, sex, and
time, and then generate region/sex/time specific posterior estimates
by applying the consistent model for the appropriate subset of the
prevalence data, together with all the rows of SMR data, and with a
minimal, defensible set of expert assumptions to fill in the remaining
model flexibility, in the case, the assumption that the remission rate
is at most .05 per person-year.

There are two empirical bayes approaches that I described in detail in
Chapter~\ref{TK}, and this model provides a good place to contrast the
results.  Fitting only prevalence data as an empirical prior is
quicker, and provides predictions of prevalence age patters for
regoins with only a few rows of data for large age groups (as well as
for regions with no rows of data).  Fitting the whole world's data
consistently for the model data provides a slightly different result,
because it incorporates the relationship between incidence,
prevalence, remission, and excess mortality from the start.

TK what mortality pattern is used by the consistent world model?

\subsection{Depression outside of North America}
12539 
16152
 The prevalence data, on the
other hand, does cover the majority of the 21 regions that partition
the world in the GBD2010 study, albeit quite non-uniformly.  There are
over 100 rows of data about each of North American High Income and
Western Europe, while there are less than 10 rows of data about Latin
America, Center; Latin America, Southern; and Sub-Saharan Africa,
Southern; and no rows of data at all about 4 other regions.

\subsection{Hepatitis C}
Diseases with strong cohort effects, like hepatitis C in North America,
violate the assumptions from Section~\ref{TK} enough to
necessitate a special approach.  That is what I investigate in this
section.

The data in Figure~\ref{hep_c-data} shows the problem: there is a peak
in prevalence for the 30-50 year olds, yet there is a strong belief
among experts that the decline in prevalence seen in 50-60 year olds
is not due to remission or excess-mortality, since there is simply not
enough of it to explain this decrease.  Instead, disease experts
believe that this age pattern is the result of a ``cohort effect'',
where a spike in incidence was caused by factors like unsafe
injections and hightened intraveneous drug use during the 1970s.  This
spike differentially affected an at-risk age group, which is now the
cohort in the cohort effect.  This group with hightened prevalence is
now aging through the population, as the 15-25 year olds of the 1970s
become the 35-45 year olds of the 1990s.  This phenomonen violates the
``time stationarity'' assumption from
Section~\ref{theory-forward_sim-compartmental_model-simplying_assumptions},
and does so to such a degree that any incidence rates generated from
this prevalence data would be completely unbelievable.  Even fitting
the prevalence data to be consistent with expert derived bounds on
remission and mortality levels is not possible; see
Figure~\ref{hep_c-consistent}.
\begin{figure}
\begin{center}
\includegraphics[width=\textwidth]{hep_c-smoothing.pdf}
\end{center}
\caption{Caption TK---TK Figure showing hep c data and fits with varying smoothness priors}
\label{hep_c-data}
\end{figure}

When the system dynamics model is removed and only the age integrating
negative binomial rate model for prevalence remains, for rows of data
with the form $(p_i, n_i, r_i, s_i, t_i, a_{0,i}, a_{1,i}, X_i,
\boldw_i)$, where $p_i$ is the measure prevalence value, $n_i$ is the
effective sample size, $r_i, s_i,$ and $t_i$ are the region covariate
(categorical), sex covariate (ordinal), and time covariates
(continuous, normalized), $(a_{0,i}, a_{i,1})$ is the age group, $X_i$
is the ``bias'' covariate, and $\boldw_i$ is the age weight function,
the full specification of the model is the following
\begin{align*}
p_i n_i &\sim \NegativeBinomial(\pi_i n_i, \delta_i)\\
\pi_i &= \int_{a_{0,i}}^{a_{1,i}} \boldpi(a; X_i)d\boldw_i(a)\\
\boldpi(a; X_i) &\sim \PCGP(\{a_1, \ldots, a_A\}; \boldmu(X_i), \scC_\rho)\\
\boldmu(a_j; X_i) &= \exp\left\{\alpha_s s_i + \alpha_t t_i + \alpha_{r_i} +
\beta X_i + \gamma_{a_j}\right\}\\
\calC_\rho &= \Matern(\nu=2, \sigma=10, \rho)\\
\alpha_s, \alpha_t, \alpha_r &\sim \Normal(0, TK)\\
\beta &\sim \Normal(0, TK)\\
\gamma &\sim \Normal(0, TK)\\
\delta &= 10^{\eta + \zeta X_i}\\
\eta &\sim N(\mu_{\log \delta}, .25^2)\\
\zeta &\sim N(0., .125^2)
\end{align*}

\begin{figure}
\begin{center}
a)
\includegraphics[width=\textwidth]{hep_c-consistent0.pdf}
b)
\includegraphics[width=\textwidth]{hep_c-consistent1.pdf}
\end{center}
\caption{Caption TK, TK figure showing the results of a consistent fit
  assuming time stationarity with no priors on remission and
  excess-mortality, with priors on remission and excess-mortality that
  experts would agree to, and the results of a fit of prevalence only.
}
\label{hep_c-consistent}
\end{figure}

In the case of the data for North America High Income, there is a
high-quality, nationally representative dataset from the NHANES study
that provides a prevalence age pattern with age groups only slightly
different from those needed for the GBD2010 study.  For this data,
visual inspection shows that the age integrating negative binomial
model for the prevalence rate suffices for estimating prevalence for
the GBD2010 age groups (Figure~\ref{hep_c-data}).  
This figure also shows the effects of varying the smoothing prior on
the piecewise constant Gaussian process.

In the case of other regions, where nationally representative data
from an NHANES-like study is not available, the model provides a way
to combine the noisy data.  Here, however, the expert priors do have
more of an effect on the results:

TK figure showing different results for a noisier region that NAHI,
possibly by varying smoothness prior and heterogeneity prior.  In the
caption, perhaps, a comparison of the dic values, showing which model
is prefered according to this metric.

Posterior predictive checks as an alternative approach to telling if
the model is doing a reasonable job.

The model reduces to the following, for a row of data collected in
systematic review, with prevalence $p_i$ and effective sample size $n_i$ for
age group $(a_{0_i}, a_{1_i})$, the rate model, age pattern model,
and age group model from Chapter~\ref{TK} are used, together with
covariates $(t_i,s_i,x_i)$ for time, sex, and ``bias'', but without any requirement to
fit together consistently with incidence, remission, or
excess-mortality, in the following mixed effects negative binomial
regression:
\begin{align*}
p_i n_i &\sim \NegativeBinomial(\pi_i n_i, \delta)\\
\pi_i &= \int _{a=a_{0_i}} ^{a=a_{1_i}} \boldpi(a) dw(a)\\
\boldpi &\sim \PLGP(\mu_\boldpi, \calC)\\
\mu_\boldpi &= \exp\left\{ t_i \alpha_t + s_i \alpha_s + x_i \beta + \boldgamma(a) \right\}\\
\alpha_t, \alpha_s &\sim \Normal(0, TK)\\
\beta &\sim \Normal(0, TK)\\
\boldgamma(a) &\sim \text{PiecewiseLinearSpline}(0, TK)
\end{align*}

The hyper-priors on $\alpha_t, \alpha_s, \beta, and \boldgamma$ are
sufficiently uninformative that changing them all by TK changes the
results by less than TK.  The uninformativity of these priors does
affect computational efficiency, however, and TK quantification of how
much time is needed to fit the less informative models.

TK Particular focus on the bias effect coefficient in the 2005 NAHI estimate.
