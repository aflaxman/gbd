\section{Introduction}
This book, \emph{TK Title} is a book-length treatment of model-based
meta-analytic methods for descriptive epidemiology.  It develops, from
first principles, the integrative systems model which constitutes the
theoretical foundation of Years Lived with Disability (YLD) estimation
in burden of disease studies like the Global Burden of Disease 2010
(GBD2010).  The estimation approach relies on Bayesian inference of
the parameters in a compartmental model of the progression of disease
through a population.  This builds on work in generic disease modeling
that has been in use for almost twenty years in global health
epidemiology \cite{Barendregt_Generic_2003}.  However, until now, the
description of the model and the method have been scattered through
the scientific literature in a loose collection of journal articles,
burden of disease reports, and operations manuals.

In addition to collecting the prior work on compartmental modeling of
disease together in one place, this book significantly extends the
model, by formally connecting the system dynamics model of disease
progression to a statistical model of epidemiological rates, the kind
that are calculated in descriptive epidemiological research and
collected together in a systematic review.  This combination of systems
dynamics modeling and statistical modeling, which I call
\emph{integrative systems modeling} allows the model to integrate all
available relevant data.  Because advanced numerical algorithms are
needed to fit these complex models, a section of the book provides the
necessary background on Markov chain Monte Carlo (MCMC) computation.

Experience with the results of systematic review indicates that when
all available relevant data is collected, it is often very
\emph{sparse} and very \emph{noisy}.  The integrative systems models
developed in this book focus particularly on techniques for handling
sparse, noisy data.  The book explores statistical models for
over-dispersed count data, covariate modeling to explain
systematic variation in epidemiological data and increase
predictive accuracy for estimates where no data is
available, and age-pattern modeling to systematically incorporate
expert knowledge about how epidemiological rates vary as a function of
age.  It also develops a novel theory of age-group modeling to address
heterogeneity in age groups commonly found during systematic review.

The theoretical foundations of integrative systems modeling of disease
in populations consititute the first half of this book.  The second
half of the book contains a series of applications of the model to the
meta-analysis of more than a dozen different diseases.  These
practical applications demonstrate how the model performs in a variety
of scenarios, and also investigate how the model performs when the
model assumptions are violated, and how these assumption violations
may be addressed.

The book concludes with a detailed description of the future
directions for research in model-based meta-analysis of descriptive
epidemiological data and integrative systems modeling for global
health.

