\chapter{Expert priors in compartments models: bipolar disorder}
\label{applications-prior_level_vals}

The meta-analysis of data from a systematic review on the descriptive
epidemiology of bipolar disorder provides an excellent example of the
effects of informative priors on levels of age-specific incidence and
remission hazards.

Bipolar disorder is a mental disorder that causes mood swings
fluctuating between euphoric highs called manic episodes and
depressive lows, interspersed by periods of residual symptoms.  Manic
episodes may last from days to months, causing personal, social and
work-related problems.  Mood swings may occur as infrequently as
yearly or as frequently as several times a day.  Extreme behavior
changes accompany mood changes, and it is not uncommon for sleeping,
eating or activity patterns to change with manic and depression
episodes.  There is no clear cause for episodes, but life changes,
medications and sleeplessness may trigger manic periods.  While there
is no cure, treatment helps manage mood swings and related
symptoms. \cite{kloos_bipolar_2011, angst_historical_2000}

The modeling of bipolar disorder is based on literature describing it
as a chronic illness with little or no complete remission.  No studies
were found reporting on complete remission from bipolar disorder,
which is equivalent to a cure rather than a temporary reduction in
symptom levels.  This is consistent with the description in the
literature that there is no cure for bipolar
disorder. \cite{american_diagnostic_2000} Figure \ref{fig:app-bipolar
  data} shows the data used for analysis.

    \begin{figure}[h]
        \begin{center}
            \includegraphics[width=\textwidth]{bipolar-data.pdf}
            \caption{Data included for the modeling of bipolar
              disorder.}
            \label{fig:app-bipolar data}
        \end{center}
    \end{figure}

\section{Prevalence and Incidence age of onset}
While there is evidence to suggest that bipolar disorder commonly
starts in the mid-teens or early twenties, there is still disagreement
over a minimum age of onset. Even though symptoms can be tracked back
to childhood, setting a threshold for diagnosis is difficult given
that current diagnostic criteria are based on adult presentation of
the disorder. Literature and expert advice suggest that although
pre-pubertal bipolar disorder is rare, there is a possibility it may
exist. \cite{kloos_bipolar_2011, angst_historical_2000} Therefore a
prior limiting age of onset to ages 10-80 was used.

    \begin{figure}[h]
        \begin{center}
            \includegraphics[width=\textwidth]{bipolar-zero_before_ten.pdf}
            \caption{Estimates of bipolar epidemiologic parameters for
              Western Europe males in 1990 using a compartmental
              model.}
            \label{fig:app-bipolar fit}
        \end{center}
    \end{figure}

While expert priors are useful in guiding the modeling process, they
may have unintended effects as discussed in Chapter
\ref{theory-expert_priors}.  Choosing to have no restrictions on the
age of onset, the age-specific prevalence differs greatly, as shown in
Figure \ref{fig:app-bipolar bounds}.

    \begin{figure}[h]
        \begin{center}
            \includegraphics[width=\textwidth]{bipolar-bounds.pdf}
            \caption{Estimates of the prevalence of bipolar disorder
              for Western Europe males in 1990 using differing priors
              that limit the age of onset in a compartmental model.}
            \label{fig:app-bipolar bounds}
        \end{center}
    \end{figure}

Like the age of onset, little is known about the upper age limit of
bipolar disorder.  Therefore a prior restricts the upper age limit to
65 years for incidence as it led to the most plausible fit to the
data.  Using expert knowledge set plausible bounds on the level of
disease is useful in modeling noisy data, but changes the upper age
limit may produce unexpected changes as shown in Figure
\ref{fig:app-bipolar onset}.

    \begin{figure}[h]
        \begin{center}
            \includegraphics[width=\textwidth]{bipolar-45_65_100.pdf}
            \caption{Estimated prevalence, incidence, remission and
              excess mortality for Western Europe males with bipolar
              disorder in 1990 using a compartmental model with
              different priors that restrict the upper age limit of
              incidence to 45, 65 or 100.}
            \label{fig:app-bipolar onset}
        \end{center}
    \end{figure}

\section{Residual v Remission}
Bipolar disorder has three health states: mania, depression and
`residual'.  Although residual symptoms can be less severe than manic
and depressive episodes, there is still considerable disability and
therefore burden.  In studies, however, concentrating on mania and
depression may lead to missing cases who are in this residual phase,
thus underestimating prevalence. \cite{angst_historical_2000}.

The terms `residual' and `remission' have very different implications
for the GBD 2010 study.  A residual state involves less severe
symptoms with lesser disability which still contribute to
burden. Remission is equivalent to cure rather than a temporary
reduction in symptom levels thus not contributing to burden. Since
there is no consistent use of these terms in the bipolar literature,
no remission data were included in the bipolar modeling. Instead,
expert guidance set a level prior on the remission rate.  The
compartmental model is sensitive to choice of prior level.  As shown
in the Figure \ref{fig:app-bipolar remission}, choice of prior leads
to large changes in the estimated excess mortality.

    \begin{figure}[h]
        \begin{center}
            \includegraphics[width=\textwidth]{bipolar-0_5_10.pdf}
            \caption{Remission (panel (a)) and Estimate excess
              mortality (panel (b)) estimates for bipolar disorder in
              Western Europe males in 1990 in a compartmental model
              with different priors on remission which limit remission
              to 0, 5, 10 per 100 PY.}
            \label{fig:app-bipolar remission}
        \end{center}
    \end{figure}

