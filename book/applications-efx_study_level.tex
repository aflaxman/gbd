\chapter{Cross-walking with fixed effects: anxiety disorders}
\label{applications-efx_study_level}
\chapterprecis{Amanda Baxter, Jed Blore, Abraham D. Flaxman, Theo Vos, and Harvey Whiteford}

The data collected in systematic review often contain a variety of
different study types or diagnostic criteria, which create systematic
biases in the measured data.  An extreme example was found in the
systematic review of diabetes prevalence, where there were $18$ variants
of diagnostic criteria.  The systematic review of anxiety disorders
provides a simpler example, which is the focus of this chapter. This
systematic review collected studies that used a handful of different
recall periods to ask about the presence of the disorders. The quantity
of interest for estimation was point prevalence, the
proportion of the population with the condition at an instant in time.
We used a fixed-effect model to adjust for the bias introduced by
studies that measured period (e.g., past-year) prevalence, since these studies also provide
valuable information on the descriptive epidemiology of the condition.
This bias adjustment by fixed-effect modeling is also called a
``cross-walk.''

Anxiety disorders include at least $8$ separate conditions each
characterized by prominent anxiety at a level that interferes with
daily life.  Not all anxiety disorders manifest in similar ways.
While generalized anxiety disorder is typically marked by persistent
worry, panic disorder is usually characterized by intense fear for
discrete periods of time. \cite{american_psychiatric_association_diagnostic_2000} As there is
a lot of comorbidity between individual anxiety disorders, anxiety
disorders were modeled together as a single condition in the GBD 2010
Study.

Anxiety disorders do not have a consistent recall period for the
measurement of epidemiological rates.  Therefore, the data from
systematic review include studies with measurements of point prevalence
and period prevalence (i.e., $6$-month or past-year prevalence).  The
analysis excludes lifetime prevalence measurements because such estimates
are particularly prone to recall bias.  Due to the nonnegligible
remission rate for anxiety disorders, period prevalence is typically
higher than point prevalence, as seen in figure~\ref{fig:app-anxiety
  data}.

    \begin{figure}[h]
        \begin{center}
            \includegraphics[width=\textwidth]{anxiety-data_by_cv.pdf}
            \caption[Systematic review data of anxiety disorders.]{A comparison of (a) point and (b) period prevalence data
              for anxiety disorders in Australasian females, collected in a systematic review for
              2000--2008.}
            \label{fig:app-anxiety data}
        \end{center}
    \end{figure}

Excluding period prevalence measurements reduces the quantity of data
and produces results that do not reflect the regional variation
present in the excluded data.  But including the period prevalence
measurements without a covariate to adjust for their systematic bias
leads to estimates that are noticeably higher in regions where there
are data on point and period prevalence.  Using a fixed effect on a
period prevalence indicator covariate allows the model to use all
available data and explain the systematic bias and variation that
result from different recall periods, as seen in
figure~\ref{fig:app-anxiety FE}.

    \begin{figure}[h]
        \begin{center}
            \includegraphics[width=\textwidth]{anxiety-FE.pdf}
            \caption[Comparison of prevalence estimates of anxiety disorders 
              using different methods.]{Comparison of prevalence estimates for anxiety
              disorders in 2005 in Australasian females using point
              prevalence data only, and using point and period prevalence data
              with and without a fixed effect.}
            \label{fig:app-anxiety FE}
        \end{center}
    \end{figure}

The results of the model with a fixed effect on recall period show
that studies on period prevalence typically measure prevalence levels
that are $49$\% (95\% UI: $[12, 91]$) higher than if they measured point
prevalence.

A limitation of applying this method to the global data set is that it
assumes the cross-walk factor is identical for all regions of the
world.  In practice, there is rarely enough data to move beyond this
assumption.  However, future applications may benefit from modeling
interactions between cross-walk covariates and age, sex, time, or
geography.
