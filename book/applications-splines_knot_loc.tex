\chapter{Knot selection in spline models: cocaine dependence}
\label{applications-splines_knot_loc}

Epidemiologic parameters, such as prevalence, have age patterns.  To provide
estimates for these age-specific hazards, many modeling decisions, must be made.
The following example of cocaine dependence illustrates the importance of
knot selection in spline models.

Symptoms associated with cocaine dependence are compulsive use and
difficulty with abstinence.  The American Psychiatric Association
recognizes cocaine dependence as fulfilling three or more of the
following seven substance dependence criteria: \cite{american_diagnostic_2000, wagner_first_2002}
    \begin{itemize} \label{page:app-substance_dependence}
        \item tolerance;
        \item withdrawal;
        \item substance is taken in larger amounts or over longer
          period than intended;
        \item persistent desire or unsuccessful efforts to control
          substance use;
        \item great deal of time is spent to obtain, use, or recover
          from effects of substance;
        \item important social, occupational, or recreational
          activities are reduced because of substance use;
        \item continued substance use despite knowledge of
          physiological or psychological problems induced by substance
          use.
    \end{itemize}

Despite a large body of data on cocaine \emph{use}, there is little
data available on the descriptive epidemiology of cocaine
dependence.\cite{Degenhardt_GBDrugs_2011} Systematic review for cocaine
dependence identified twenty-eight rows of prevalence data,
covering 3 regions.  For this example, we have restricted our attention
to only data from the United States of America (Figure \ref{fig:app-cocaine_data}).

    \begin{figure}[h]
        \begin{center}
            \includegraphics[width=\textwidth]{cocaine_dependence-data.pdf}
            \caption{Prevalence data for cocaine dependence in the United States of America.}
            \label{fig:app-cocaine_data}
        \end{center}
    \end{figure}

As discussed in Chapter \ref{theory-age_pattern_model}, we model
age-specific hazards with spline models.  To be more specific, in this
case, the spline model takes the form of a continuous, piecewise
linear function, with ``knots'' where the function is non-linear
selected as part of the model.  These knots partition the age range
into intervals, and the choice of knots can be influential on the
resulting estimates.  In a setting where data is \emph{not} sparse and
noisy, estimates will not be very sensitive to choice of knots.
However, when working with sparse and noisy data, the number and
location of knots are important decisions as they can influence the
model results substantially.  Thus, the number of knots and locations
should be chosen a priori using expert knowledge concerning the
disease being modeled when the data are sparse and noisy.  It is also
important to consider additional knots and alternative configurations
of knots as a sensitivity analysis.

To demonstrate the importance of the number of knots in a spline
model, three models with differing numbers of knots are compared in
Figure \ref{fig:app-cocaine_knots}.  The original model for cocaine
dependence has knots at 0, 13, 17, 22, 30, 35, 40, 45, 50, 60, and
100.  The `less knot' model limits knots to ages 0, 13, 25, 65, and
100.  During the critical period of 13-50 years, the `more knot' model
has knots every three years \{13, 16, 19, \ldots, 43, 46, 49\} with
additional knots at \{0, 55, 60, 80, 100\}.  As seen in Figure
\ref{fig:app-cocaine_knots}, choosing too few or too many knots can
influence the model substantially.

    \begin{figure}[h]
        \begin{center}
            \includegraphics[width=\textwidth]{applications/cocaine_dependence-knots.pdf}
            \caption{Prevalence estimates of cocaine dependence in the United
              States of Americausing a single rate type model in the original
              model and models with less and more knots. }
        \label{fig:app-cocaine_knots}
        \end{center}
    \end{figure}

A penalized spline model with a smoothing parameter is another
solution to knot selection.  The model includes more knots but adds a
penalty to discourage the model from using more knots than necessary
for the data.  The smoothing parameter controls the roughness of the
estimating function and the fidelity of the estimates to the data,
making knot selection less influential as shown in Figure
\ref{fig:app-cocaine_smoothing}.  However, too much smoothing leads
to overcompression of the prevalence estimates with
estimates that are not representative of the data.

    \begin{figure}[h]
        \begin{center}
            \includegraphics[width=\textwidth]{applications/cocaine_dependence-smoothing.pdf}
            \caption{Prevalence estimates from the original model
              using a penalized spline model with a smoothing
              parameter $\sigma$. }
        \label{fig:app-cocaine_smoothing}
        \end{center}
    \end{figure}

As the result of modeling age-specific hazards with piecewise
continuous splines, knot selection can be an influential part of the
model.  From the sensitivity analysis in Figure \ref{fig:app-cocaine_knots},
it is clear that there is not much benefit from adding additional
knots in the model, especially since there is available data.  As seen
in the next chapter, a sensitivity analysis is crucial to determine
the model's sensitivity to knot selection when data is sparse and noisy.
