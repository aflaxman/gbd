\chapter{Knot selection in spline models: cannabis dependence}
\label{applications-splines_knot_loc}

The meta-analysis of data from systematic review of the prevalence of
cannabis dependence has a clear age-pattern, providing an example of
the importance of knot selection in spline models.

Symptoms associated with cannabis dependence are compulsive use and
difficulty with abstinence.  The American Psychiatric Association
recognizes cannabis dependence as fulfilling three or more of the
following seven substance dependence criteria:
    \begin{itemize} \label{page:app-substance_dependence}
        \item tolerance
        \item withdrawal
        \item substance is taken in larger amounts or over longer
          period than intended
        \item persistent desire or unsuccessful efforts to control
          substance use
        \item great deal of time is spent to obtain use or recover
          from effects of substance
        \item important social, occupational or recreational
          activities are reduced because of substance use
        \item continued substance use despite knowledge of
          physiological or psychological problems induced by substance
          use \cite{american_diagnostic_2000, coffey_cannabis_2002}
    \end{itemize}

There is little data available on cannabis dependence.  Fifty-three
rows of data were identified for cannabis dependence prevalence,
covering 3 regions (Figure \ref{fig:app-cannabis_data}).

    \begin{figure}[h]
        \begin{center}
            \includegraphics[width=\textwidth]{cannabis_dependence-data.pdf}
            \caption{Global prevalence data for cannabis dependence.}
            \label{fig:app-cannabis_data}
        \end{center}
    \end{figure}

As discussed in Chapter \ref{theory-age_pattern_model}, age-specific
hazards are modeled with spline models.  A spline model is any
piecewise polynomial function.  Knots partition the age range into
intervals.  With ample data and clear age patterns, models will not be
very sensitive to choice of knots.  However, when working with sparse
and noisy data, the number and location of knots are important
decisions as they can influence the model results substantially.
Thus, the number of knots and locations should be chosen a priori
using expert knowledge concerning the disease being modeled when the
data are sparse and noisy.

To demonstrate the importance of the number of knots in a spline
model, three models with differing numbers of knots are compared in
Figure \ref{fig:app-cannabis_knots}.  The original model for cannabis
dependence has knots at 0, 13, 17, 22, 30, 35, 40, 45, 50, 60, and
100.  The `less knot' model limits knots to ages 0, 13, 25, 65, and
100.  During the critical period of 13-50 years, the `more knot' model
has knots every three years \{0, 13, 16, 19, \ldots, 43, 46, 49\} with
additional knots at \{55, 60, 80, 100\}.  As seen in Figure
\ref{fig:app-cannabis_knots}, choosing too few or too many knots can
influence the model substantially.

    \begin{figure}[h]
        \begin{center}
            \includegraphics[width=\textwidth]{applications/cannabis_dependence-knots.pdf}
            \caption{Prevalence estimates of cannabis dependence using
              a single rate type model in the original model and
              models with less and more knots. }
        \label{fig:app-cannabis_knots}
        \end{center}
    \end{figure}

A penalized spline model with a smoothing parameter is another
solution to knot selection.  The model includes more knots but adds a
penalty to discourage the model from using more knots than necessary
for the data.  The smoothing parameter controls the roughness of the
estimating function and the fidelity of the data, making knot
selection less influential as shown in Figure
\ref{fig:app-cannabis_smoothing}.  However, too much smoothing and the
overcompression of the prevalence estimates is not representative of
the data.

    \begin{figure}[h]
        \begin{center}
            \includegraphics[width=\textwidth]{applications/cannabis_dependence-smoothing.pdf}
            \caption{Prevalence estimates from the original model
              using a penalized spline model with a smoothing
              parameter $\sigma$. }
        \label{fig:app-cannabis_smoothing}
        \end{center}
    \end{figure}

