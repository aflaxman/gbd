\chapter{Empirical priors: pancreatitis}
\label{applications-priors_empirical}

Systematic review for the GBD 2010 Study often found a few regions for
which detailed data on the age pattern of disease was available, but
many more regions where cases were reported much sparser
age-specificity.  Hierarchical modeling using an empirical Bayesian
prior is our way to conduct partial pooling, and borrow strength from
the regions with age-specific data to produce estimates of age
patterns in regions where little or no age-specific data is available.
This chapter demonstrates how the partial pooling functions in a
setting where it is quite pronounced, in estimating sex-specific
pancreatitis incidence in Eastern Europe.

Pancreatitis is the inflammation of the pancreas, most commonly
caused by alcohol or gallstones.  In most cases, the disease resolves
itself and there is no need for treatment.  However, some acute
cases develop pancreatic necrosis and systemic organ failure.  These
complications require immediate treatment and have a high mortality risk.
\cite{raraty_acute_2004, banks_epidemiology_2002, sekimoto_JPN_2006}

Data from systematic review yielded $3950$ incidence data points,
$373$ of which were from Eastern Europe, which constitute the example
in this chapter.  As shown in Figure~\ref{fig:app-pan data}, the data
are very heterogeneous.

    \begin{figure}[h]
        \begin{center}
            \includegraphics[width=\textwidth]{pancreatitis-data.pdf}
            \caption{Pancreatitis incidence data
              with pooled estimates for males and females in Eastern Europe in 2005.}
            \label{fig:app-pan data}
        \end{center}
    \end{figure}

Closer investigation shows that some of the heterogeneity in the data
is caused by different age patterns in males and females.  The
age-specific incidence rate of pancreatitis has a peak around age 40
in males, but not in females.  Using the pooled data estimates from
Figure~\ref{fig:app-pan data} as an empirical prior for estimates
based only on sex-specific data, posterior estimates for males and
females with partial pooling are shown in Figure~\ref{fig:app-pan
  compare}.

    \begin{figure}[h]
        \begin{center}
            \includegraphics[width=\textwidth]{pancreatitis-compare.pdf}
            \caption{A comparison of incidence estimates for males (a) and
              females (b) in Eastern Europe in 2005.  The estimated incidence 
              using pooled data from Figure~\ref{fig:app-pan data} was applied 
              as an empirical prior to the sex-specific incidence shown to 
              improve estimates.}
            \label{fig:app-pan compare}
        \end{center}
    \end{figure}

Empirical priors provide data-derived relationships to guide the
modeling process.  In cases where the data is less clear, empirical
priors provide a principled and computationally tractable way to
borrow strength between regions for estimation.
