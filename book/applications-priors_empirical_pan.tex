\chapter{Empirical priors: pancreatitis}
\label{applications-priors_empirical}
\chapterprecis{David Chou, Hannah M. Peterson, Abraham D. Flaxman, Christopher J. L. Murray, and Mohsen Nagavi}

Systematic review for GBD 2010 often found a few regions for
which detailed data on the age pattern of disease were available, but
many more regions for which cases were reported with much sparser
age specificity.  Hierarchical modeling using an empirical Bayesian
prior is our way to conduct partial pooling and borrow strength from
the regions with age-specific data to produce estimates of age
patterns in regions where few or no age-specific data are
available.  This chapter demonstrates the results of partial pooling
at the regional level, where country-to-country variation is quite
pronounced, by examining the estimation of age-specific pancreatitis
incidence in Western Europe.

Pancreatitis is the inflammation of the pancreas, most commonly
caused by alcohol or gallstones.  In most cases, the disease resolves
itself and there is no need for treatment.  However, some acute
cases develop pancreatic necrosis and systemic organ failure.  These
complications require immediate treatment and have a high mortality risk.
\cite{raraty_acute_2004, banks_epidemiology_2002, sekimoto_jpn_2006}

Data from systematic review yielded $3950$ incidence data points,
$1053$ of which were from Western Europe and constitute the example
in this chapter.  As shown in figure~\ref{fig:app-pan data}, the data
from Western Europe are very heterogeneous and the pooled estimates do
no reflect the data the best, especially between the ages of $25$--$60$.

    \begin{figure}[h]
        \begin{center}
            \includegraphics[width=\textwidth]{pancreatitis-we_data.pdf}
            \caption[Systematic review data for pancreatitis with estimates.]{Pancreatitis incidence data with pooled estimates for males and females in Western
            Europe in 2005.}
            \label{fig:app-pan data}
        \end{center}
    \end{figure}

Closer investigation in figure~\ref{fig:app-pan compare} shows that there is
considerable heterogeneity caused by different age patterns between countries.
The pooled estimate in figure~\ref{fig:app-pan data} (dashed line,
figure~\ref{fig:app-pan data}) does not capture this variation.  However,
country-specific posterior estimates can be made with an empirical prior and
partial pooling, as shown in figure~\ref{fig:app-pan compare}.

Empirical priors have the benefit of coping with large differences in age-specific
rates and borrowing strength between countries.  The empirical prior is the
estimate from the pooled regional data from figure~\ref{fig:app-pan data}.  A
posterior distribution for each country can is estimated with this empirical prior.
With lots of data, the empirical prior is irrelevant, as the data informs the
posterior estimate.  When there are no data,
the posterior estimate follows the empirical prior with large
uncertainty intervals since the countries with data show a lot of
country-to-country variation (this large uncertainty is also the
reason why the means of the empirical prior and posterior estimates for Germany
in panel (d) do not match precisely).  With some data, the the posterior finds the
proper balance between the data and the empirical prior.

    \begin{figure}[h]
        \begin{center}
            \includegraphics[width=\textwidth]{pancreatitis-we_compare.pdf}
            \caption[Comparison of pancreatitis incidence estimates.]
             {Comparison of pancreatitis incidence estimates
              for males in 2005 for (a) Finland, (b) the Netherlands, (c)
              Cyprus, and (d) Germany.  The estimated incidence using
              pooled data from figure~\ref{fig:app-pan data} was
              applied as an empirical prior to the sex-specific
              incidence to improve estimates.}
            \label{fig:app-pan compare}
        \end{center}
    \end{figure}

Empirical priors provide data-derived relationships to guide the
modeling process.  In cases where the data are less clear, empirical
priors provide a principled and computationally tractable way to
borrow strength between regions for estimation. 