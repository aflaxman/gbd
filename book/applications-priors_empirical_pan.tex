\chapter{Empirical priors: pancreatitis}
\label{applications-priors_empirical}

Systematic review for the GBD 2010 Study often found a few regions for
which detailed data on the age pattern of disease were available, but
many more regions for which cases were reported with much sparser
age-specificity.  Hierarchical modeling using an empirical Bayesian
prior is our way to conduct partial pooling, and borrow strength from
the regions with age-specific data to produce estimates of age
patterns in regions where little or no age-specific data are available.
This chapter demonstrates how partial pooling functions at a regional level
where country differences are quite pronounced in the estimation of age-specific
pancreatitis incidence in Western Europe.

Pancreatitis is the inflammation of the pancreas, most commonly
caused by alcohol or gallstones.  In most cases, the disease resolves
itself and there is no need for treatment.  However, some acute
cases develop pancreatic necrosis and systemic organ failure.  These
complications require immediate treatment and have a high mortality risk.
\cite{raraty_acute_2004, banks_epidemiology_2002, sekimoto_JPN_2006}

Data from systematic review yielded $3950$ incidence data points,
$1053$ of which were from Western Europe, which constitute the example
in this chapter.  As shown in Figure~\ref{fig:app-pan data}, the data
from Western Europe are very heterogeneous.

    \begin{figure}[h]
        \begin{center}
            \includegraphics[width=\textwidth]{pancreatitis-we_data.pdf}
            \caption{Pancreatitis incidence data
              with pooled estimates for males and females in Western Europe in 2005.}
            \label{fig:app-pan data}
        \end{center}
    \end{figure}

Closer investigation shows that some of the heterogeneity in the data
is caused by different age patterns between countries.  
Using the pooled data estimates from
Figure~\ref{fig:app-pan data} as an empirical prior for estimates
based only on country-specific data, posterior estimates for countries
with partial pooling are shown in Figure~\ref{fig:app-pan compare}.  
Unless there are data to inform otherwise, the empirical
prior is assumed to be true.  It is worth noting that the empirical 
prior and posterior estimates for Germany in Figure~\ref{fig:app-pan compare} panel (d) 
do not match.  This is because Germany has a large uncertainty interval.  
In cases with small uncertainty, the posterior will more closely 
follow the empirical prior.

    \begin{figure}[h]
        \begin{center}
            \includegraphics[width=\textwidth]{pancreatitis-we_compare.pdf}
            \caption{A comparison of incidence estimates for (a) Finland,
              (b) the Netherlands, (c) Cyprus and (d) Germany in 2005.  The estimated incidence
              using pooled data from Figure~\ref{fig:app-pan data} was applied
              as an empirical prior to the sex-specific incidence to
              improve estimates.}
            \label{fig:app-pan compare}
        \end{center}
    \end{figure}
    
Empirical priors provide data-derived relationships to guide the
modeling process.  In cases where the data is less clear, empirical
priors provide a principled and computationally tractable way to
borrow strength between regions for estimation.
